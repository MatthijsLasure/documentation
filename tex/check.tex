% Getting into it: keeping taps on the simulation

There are a few tools available to keep an eye on your simulation, something 
that is strongly recommended to do.


\section{HTOP}
Most Linux distributions come with \verb|htop| preinstalled. After startup, you 
can filter on username by pressing 'u', and apply a filter by pressing 'F4'. 

When starting a LJ simulation, 8 cores will be used by Gaussian for the 
calculation of the partial charges. Afterwards, the program will run on one 
core, which should reach 100\%. Due to the OpenMP design there will be multiple 
threads visible (as set in the run script), but only one will actually do 
something.

In the case of a QM simulation, multiple cores will be used. Currently 
(27/07/2016) 2 cores will be maxed out while the other 6 hover around 5\%. This 
is normal behaviour.

If the program is running (= is present in the list of open programs) but there 
is no CPU activity it may have locked itself. Consult the logfile and the 
troubleshooting section of this manual.

On 8-core systems, a system load of 4 to 5 is observed.

\section{check.sh}
The \verb|check.sh| script is a simple wrapper for the 'tail' command: it will 
find the latest output file and display the latest line. After sleeping 1 
second it erases the line and starts over.

Start it using following command. Terminate with CTRL-C.
\begin{lstlisting}[caption=The check utility]
./check.sh log.txt
\end{lstlisting}