% Getting converted: the ConvertDump utility

The ConvertDump utility is mainly designed for converting dump- and boxfiles to 
XYZ files that can be viewed using Jmol. It can also keep track of distances, 
and sort the solvent molecules (see also chapter~\ref{sec:conf2xyz}). Following 
modes are available:

\begin{tabular}{cl}
	Mode & Description \\ \hline
	0 & Convert a dump file to XYZ \\
	1 & Convert a box file to XYZ \\
	2 & Calculate minimal distance per iteration in a dump file \\
	3 & Convert a box file to XYZ and sort solvent on distance to solute
\end{tabular}

\section{Modes}
\subsection{Dump to XYZ}
This mode generates an animated XYZ file from the dump file. Jmol can view this 
frame by frame or as a video.

NOTE: be aware that larger dump files will generate enormous XYZ files and may 
take a while to finish.

\subsection{Box to XYZ}
This mode converts a box file to a XYZ file, for viewing using Jmol.

\subsection{Distance per iteration}
Mode 2 calculates the minimal distance between one atom of the solute and one 
atom per solvent molecule. This can be useful when monitoring a hydrogen bond. 
The solute atom is then an acidic proton, and the solvent atom is the DMSO 
oxygen. In this way it is very easy to see if a hydrogen bond is formed.

\subsection{Sort}
The last mode is identical to the second one with one difference. It will sort 
the solvent molecules from closest to the solute to furthest. The distance for 
one solvent molecule is determined to be the lowest solventatom - soluteatom 
distance.

\section{Arguments}
If the program is run without arguments, the help is displayed.

\begin{itemize}
	\item \verb|-m <mode> / --mode <mode>|: determines which mode to use;
	\item \verb|-b <box> / --box <box>| (optional): path to box file. If not 
	present, the program will use STDIN for the box file;
	\item \verb|-o <output> / --output <output>| (optional): path to the output 
	variable. If omitted, the program will use STDOUT;
	\item \verb|-d <DMSO> / --dmso <DMSO>| (optional): path to the DMSO file. 
	If not given, the program will use the internal coordinates;
	\item \verb|-s <solute> / --solute <solute>|: for converting dump files 
	when no box is available;
	\item \verb|-D <dump> / --dump <dump>| (optional): path to the dump file. 
	When empty the program will use STDIN;
	\item \verb|-q / --quiet| (optional): do not produce any output. If no 
	output file is 
	specified, it will still print the output to STDOUT;
	\item \verb|n <1> <2> / --atoms <1> <2>|: the atomnumbers used for mode 2;
	\item \verb|-h / --hydrogens| (optional): when printing XYZ output, also 
	print the hydrogen atoms.
\end{itemize}

\chapter{BoxScale}
The BoxScale utility has only one use: scale a box with a given factor. Both 
the box length and the internal coordinates are multiplied with the factor, and 
then written to the output file.

\begin{lstlisting}[caption=BoxScale]
	./BoxScale <factor> <input> <output>
\end{lstlisting}