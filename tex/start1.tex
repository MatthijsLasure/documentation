% Getting settled: download, compile and set up
To use the simulation, following tools and programs are required:
\begin{itemize}
	\item Gaussian 09: for preparing, calculating spectra, calculating partial 
	charges with Lennard-Jones. If no LJ calculations are performed, Gaussian 
	does not need to be installed on the machine.
	\item MOPAC 2016: for the PM6 calculations in the QM section.
	\item A Fortran compiler: to compile the code. Works very good with 
	gfortran, altough the Intel compiler should work.
	\item MatLab: not required, but useful for analysis of the simulation and 
	viewing the spectra.
	\item Linux: The program should work with any Linux distribution. Windows 
	or Mac OS is not supported.
\end{itemize}

\section{External tools}
\subsection{Gaussian 09}
During development, Gaussian versions 09 A02 and 09D have been used, and are 
proven to work correctly.
To install Gaussian, follow the instructions. Make sure that the installation 
path is added to the \verb|$PATH|-variable. If Gaussian can be started with the 
command \verb|g09|, the installation was succesfull.

\subsection{MOPAC 2016}
MOPAC is a little tougher. After downloading, installing and activating, a 
symbolic link must be made. Issue following command in the installation 
directory:

\begin{lstlisting}[caption=Symlink to MOPAC]
ln -s MOPAC2016.exe mopac
\end{lstlisting}

Finally, make sure that the MOPAC directory is also included in the 
\verb|$PATH|-variable. The easiest way to achieve this, is by adding following 
to your \verb|.bashrc|-file:
\begin{lstlisting}[caption=changing the path variable]
export PATH=$PATH:/path/to/mopac

# Then, in the shell execute: (or restart the terminal)
source .bashrc
\end{lstlisting}

It works when you can successfully execute \verb|mopac| on the command line.

\section{Downloading and compiling}

\subsection{Downloading}
At \url{https://github.com/MatthijsLasure/MonteCarlo/releases/latest}, the 
latest release version of the program can be obtained. These versions are 
deemed to be working correctly.
Alternatively, the very last revision can be obtained at 
\url{https://github.com/MatthijsLasure/MonteCarlo/archive/master.zip}.
Please notice that this revision is not a release, and may contain bugs that 
are not noticed or in the process of being fixed. If git is installed on the 
system, following command can also be issued to obtain a ready to go directory:

\begin{lstlisting}[caption=Clone with git]
git clone https://github.com/MatthijsLasure/MonteCarlo.git
\end{lstlisting}

This command will generate a MonteCarlo-directory, containing all source files.

\subsection{Compiling}
Compiling is very easy due to the makefiles.
Inside the MonteCarlo source directory, make a directory called either 
\verb|tmp-gnu| or \verb|tmp-intel|, and issue following command:

\begin{lstlisting}[caption=Compiling the main program]
make GNU   # To use the GFortran compiler
make intel # To use the Intel compiler
\end{lstlisting}

The executable \verb|MonteCarloGNU| or \verb|MonteCarloOpenMP| will be created.
Both have the OpenMP standard enabled to allow for multi-threaded 
PM6-calculations.

All the utilities can be compiled, just by issuing the \verb|make| command in 
their respective source directories. GFortran is always used.

\section{Set-up}
It is strongly advised to make a directory for every simulation, as to keep 
files separated. Copy all necessary files into this directory, and make sure 
that the executable files (\verb|run.sh| and \verb|MonteCarloGNU|) have the 
correct permissions (executable). Following files are necessary:

\begin{tabular}{l}
	General \\ \hline
	MonteCarloGNU \\
	run.sh \\
	box.txt \\
\end{tabular}
\hspace{0.5cm}
\begin{tabular}{l}
	Lennard-Jones \\ \hline
	param.txt \\
	par\_sol.txt \\
	\null \\
\end{tabular}

The DMSO-molecule is build into the program, so a file is not necessary.
If a custom molecule is necessary, you have to provide DMSO.txt.

For a breakdown of the files and their syntax, please see chapter XX.