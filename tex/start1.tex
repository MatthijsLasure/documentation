% Getting settled: download, compile and set up
To use the simulation, following tools and programs are required:
\begin{itemize}
	\item Gaussian 09: for preparing, calculating spectra, calculating partial 
	charges with Lennard-Jones. If no LJ calculations are performed, Gaussian 
	does not need to be installed on the machine.
	\item MOPAC 2016: for the PM6 calculations in the QM section.
	\item A Fortran compiler: to compile the code. Works very good with 
	gfortran, altough the Intel compiler should work.
	\item MatLab: not required, but usefull for analysis of the simulation and 
	viewing the spectra.
	\item Linux: The program should work with any Linux distribution. Windows 
	or Mac OS is not supported.
\end{itemize}

\section{External tools}
To install Gaussian, follow the instructions. Make sure that the installation 
path is added to the \verb|$PATH|-variable. If Gaussian can be started with the 
command \verb|g09|, the installation was succesfull.

MOPAC is a little tougher. After downloading, installing and activating, a 
symlink must be made. Issue following command in the installation directory:

\begin{lstlisting}[caption=Symlink to MOPAC]
ln -s MOPAC2016.exe mopac
\end{lstlisting}

Finally, make sure that the MOPAC directory is also included in the 
\verb|$PATH|-variable. The easiest way to achieve this, is by adding following 
to your \verb|.bashrc|-file:
\begin{lstlisting}[caption=changing the path variable]
export PATH=$PATH:/path/to/mopac

# Then, in the shell execute: (or restart the terminal)
source .bashrc
\end{lstlisting}

It works when you can succesfully execute \verb|mopac| on the command line.

\section{Downloading and compiling}
At \url{https://github.com/MatthijsLasure/MonteCarlo/releases/latest}, the 
latest release version of the program can be obtained. These versions are 
deemed to be working correctly.
Alternatively, the hot last version can be obtained at 
\url{https://github.com/MatthijsLasure/MonteCarlo/archive/master.zip}