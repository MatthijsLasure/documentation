% Getting ready: preparations
Before the actual simulation can start, some preparations have to be performed.
This consists of preparing the solute, the box and performing a minimization 
using Lennard-Jones. Finally, the box is rescaled to reflect the correct 
density.

\section{Solute preparation}
This section can be performed with a program of choice that can handle 
molecules, such as ADF, Gaussian or Chem3D. Draw the molecule and quickly 
minimize (if available). Afterwards, perform a good optimisation to get a good 
structure. Good results have been achieved using Gaussian and the B3LYP/6-31G* 
potential-base set combination.
When finished, obtain the carthesian coordinates from the output. Perform a PM6 
single point calculation, and obtain the energy in Hartree.

 Open the box file, and look for the line that says \verb|SOLUTE|. After this 
 line, write the following:

\begin{enumerate}
	\item The number of solute atoms;
	\item A description of the solute (e.g. name);
	\item All atom symbols and coordinates, one line per atom;
	\item The energy in hartree;
	\item The number of bonds used for internal rotation;
	\item A list of bonds used for rotation (see below);
	\item An empty line.
\end{enumerate}

\subsection{Bonds for rotation}
The syntax for this list is unfortunatly very picky due to programming 
limitations. 
\verb|III_III_FFFFFF|