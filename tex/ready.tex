% Getting ready: preparations
Before the actual simulation can start, some preparations have to be performed.
This consists of preparing the solute, the box and performing a minimization 
using Lennard-Jones. Finally, the box is rescaled to reflect the correct 
density.

\section{IMPORTANT NOTES}
\subsection{Floating Point Values}
If the program expects a floating point value but an integer is given (because 
the value happens to be round), it will be incorrectly read. Always make sure 
there is a point.

For example, see the solute rotation setting:\\
Wrong: \vispace\vispace 8 \vispace\vispace\vispace 1 \vispace 60 \\
Good: \vispace\vispace 8 \vispace\vispace\vispace 1 \vispace 60.0

\subsection{End Of File}
Fortran expects an empty line at the end of the file. Not complying will result 
in an incorrect read. This is very important when preparing data on a Windows 
machine. It is also advised to use the \verb|dos2unix| and \verb|checknprep| 
(included in the source code) utilities to ensure correct conventions.

\section{Box preparation}
For smaller molecules ($N < 25$), the included box can be used. It contains 30 
DMSO molecules (random placed and rotated) in a \SI{20}{\angstrom} sided box.
To generate a new box, it generally suffices to generate 6 random values per 
DMSO molecule, where the first 3 represent the position of the Centre of Mass 
(CoM), and the latter are the angles of rotation. The coordinates are relative 
to the middle of the box, so they will be in $[-r/2, +r/2]$. The angles lie in 
$[-\pi, +\pi]$. Use following syntax:

\begin{enumerate}
	\item The half length of the edge of the box, in angstrom ($r/2$);
	\item Number of DMSO molecules;
	\item The energy of the box (see note);
	\item The coordinates and angles of all molecules, one per line (X Y Z 
	$\alpha$ $\beta$ $\gamma$);
	\item A line that says \verb|SOLUTE|;
	\item The solute (see next section).
\end{enumerate}

Note: the energy of the box will be calculated during the simulation when QM 
calculations are performed (see Getting Started). When starting with 
minimisations it is sufficient to set it to $0.0$.

\section{Solute preparation}
This section can be performed with a program of choice that can handle 
molecules, such as ADF, Gaussian or Chem3D. Draw the molecule and use a quick 
minimize (if available). Afterwards, perform a good optimisation to get a good 
structure. Good results have been achieved using Gaussian and the B3LYP/6-31G* 
potential-base set combination.
When finished, obtain the cartesian coordinates from the output. Perform a PM6 
single point calculation, and obtain the energy in Hartree.

 Open the box file, and look for the line that says \verb|SOLUTE|. After this 
 line, write the following:

\begin{enumerate}
	\item The number of solute atoms;
	\item A description of the solute (e.g. name);
	\item All atom symbols and coordinates, one line per atom;
	\item The energy in Hartree;
	\item The number of bonds used for internal rotation;
	\item A list of bonds used for rotation (see below);
	\item An empty line.
\end{enumerate}

\subsection{Bonds for solute rotation}
This list contains bonds that can be rotated around. The syntax is the 
following:

\verb|III_III_FFFFFF|

With III being the number for the atoms of the bond. Notice that one integer 
MUST occupy 3 characters. The underscore is a character that will be ignored 
(it is recommended to use a space), 
and FFFFF is a floating point value, that is used to indicate an individual 
maximal rotation, overriding the general maximal rotation. This is optional, 
but will be written by the program in the result file.

In short, writing the following will result in unexpected results:

\verb|8 1| will read in as $(81, 0)$ instead of $(8, 1)$. Rather use 
\verb|008 001|.

\begin{lstlisting}[caption=Example of a correct box-file]
7.7838685667999998     
30
-796.46817126240921     
BOX 9
-4.93  6.07  2.42 -2.07 -1.17   1.32     
4.66 -1.67  6.16  2.76  2.64  -1.68     
...     
SOLUTE
11
Solute: (R)-2-Chloropropanoic acid
C  0.30  0.37 -0.51
H  0.50 -0.02 -1.51
Cl 1.54 -0.39  0.57
C  0.42  1.89 -0.48
H -0.31  2.33 -1.15
H  0.25  2.27  0.52
H  1.42  2.19 -0.81
C -1.08 -0.12 -0.12
O -2.09  0.12 -0.73
O -1.05 -0.85  1.01
H -1.97 -1.13  1.18
-0.16947515     
1
8   1  60.000000

\end{lstlisting}

It cannot be stressed enough that there MUST be an empty line at the end of the 
file to ensure correct reading of the data.

\section{Minimisation theory}
In the run script, there is a section for the minimisation, between the 
\verb|if false; then| en \verb|fi| statements. To run both the minimisation and 
the simulation, change the \verb|false| to \verb|true|. If you only want to run 
the minimisation, copy the first section (until the \verb|fi|) to a seperate 
run script.

The run script will perform 4 Lennard-Jones simulations with 1 million steps 
each. In between simulations the box is rescaled.

The density of the box can be calculated using formula~\ref{eqn:density}. If 
assumed that the solute is of equal volume as a DMSO molecule (which is the 
case for e.g. chloropropanoic acid), n is 31 and the density is $0,5027 
\textrm{ g cm}^{-3}$. After a rescaling the correct edge length is obtained to 
get the density of $1,09037 \textrm{ g cm}^{-3}$.\cite{Radhamma2008}
The process is done in multiple steps to allow the system to relax so no 
molecules are clipping into each other.

\begin{equation} \label{eqn:density}
\rho = \dfrac{n}{(r \si{\angstrom})^3} = \dfrac{n}{(r \si{\cdot 10^{-7} cm})^3} 
=
\dfrac{n}{(r \cdot 10^{-7})^3} \cdot \dfrac{M_{DMSO}}{N_A} \dfrac{g}{cm^3}
\end{equation}

After the LJ simulations, a long QM simulation is run to ensure the box is well 
optimized.

In the config file (\verb|config0.ini|), the parameters have been set to not 
rotate the solute (as 
this is not the actual simulation). As explained in section XX, the number of 
steps for LJ and GA are ignored as they are passed via command line arguments.

The minimazation is done in several steps in the run script. If necessary, 
change the scaling factor to obtain a bigger or smaller box.

% Getting started: the simulation

\section{Checking the output}
When the minimization is done, open the last log file and ensure that the 
energy is negative. For chloropropanoic acid, typical values are around $-800 
\textrm {kJ/mol}$.

If the energy is positive some molecules are still clipped into each other. 
This can be verified by visualisation using the ConvertDump tool and Jmol (see 
chapter 
\ref{chap:convertdump}).
In this case make another long LJ run (possibly use 5 million), again followed 
by another long GA run.

\section{Starting}
If you already did the minimisation and choose for the all-in-one package, you 
don't have to do anything.

If you are starting without minimisation, make sure it's turned off in the 
\verb|run.sh| script. Then start the script.

\subsection{Where to run?}
The program is likely to run very long. It is thereby very important to check 
how to start the program: if you run it in a SSH session, it will be terminated 
when the session closes due to network issues or a client crash.

To work around this problem, one can use the 'byobu' package. This will open a 
new shell that stays alive, even when the calling terminal is closed. 

Open a new shell using \verb|byobu bash|. To detach the shell, press CTRL-A and 
then CTRL-D. For reattaching the shell, just type \verb|byobu|.


\subsection{Start}\label{subs:start}
When you are ready to start, run following command:

\begin{lstlisting}[caption=runmin]
./run.sh &> log.txt
\end{lstlisting}

By using \verb|&> log.txt|, all messages are written to log.txt.

Congratulations! The program should be running now!