	\chapter{Getting settled: download, compile and set up}
	% !TeX spellcheck = en_GB
	% Getting settled: download, compile and set up
	To use the simulation, following tools and programs are required:
	\begin{itemize}
		\item Gaussian 09: for preparing, calculating spectra, calculating 
		partial 
		charges with Lennard-Jones. If no LJ calculations are performed, 
		Gaussian 
		does not need to be installed on the machine.
		\item MOPAC 2016: for the PM6 calculations in the QM section.
		\item A Fortran compiler: to compile the code. Works very good with 
		gfortran, altough the Intel compiler should work.
		\item MatLab: not required, but useful for analysis of the simulation 
		and 
		viewing the spectra.
		\item Linux: The program should work with any Linux distribution. 
		Windows 
		or Mac OS is not supported.
	\end{itemize}
	
	To compile the source code, you need the \verb|make| utility and a 
	compiler. 
	Following compilers have been successfully used. Other versions are likely 
	to 
	work, but are untested. To enable parallelization, the compiler must 
	support 
	OpenMP.
	\begin{itemize}
		\item Intel Fortran Compiler version 16.0.3
		\item GNU Fortran Compiler versions 4.6.3 and 4.8.5
	\end{itemize}
	
	\section{External tools}
	\subsection{Gaussian 09}
	During development, Gaussian 09\cite{g09} versions A02 and 09D have been 
	used, 
	and are proven to work correctly.
	To install Gaussian, follow the instructions. Make sure that the 
	installation 
	path is added to the \verb|$PATH|-variable. If Gaussian can be started with 
	the 
	command \verb|g09|, the installation was succesfull.
	
	\subsection{MOPAC 2016}
	MOPAC\cite{mopac} is a little tougher. After downloading, installing and 
	activating, a 
	symbolic link must be made. Issue following command in the installation 
	directory:
	
	\begin{lstlisting}[caption=Symlink to MOPAC]
	ln -s MOPAC2016.exe mopac
	\end{lstlisting}
	
	Finally, make sure that the MOPAC directory is also included in the 
	\verb|$PATH|-variable. The easiest way to achieve this, is by adding 
	following 
	to your \verb|.bashrc|-file:
	\begin{lstlisting}[caption=changing the path variable]
	export PATH=$PATH:/path/to/mopac
	
	# Then execute: (or restart the terminal)
	source .bashrc
	\end{lstlisting}
	
	The installation is succesfull when executing \verb|mopac| returns the 
	welcome 
	message. Don't forget to activate the program!
	
	\section{Downloading and compiling}
	
	\subsection{Downloading}
	The source code can be obtained at the links below. The first link refers 
	to 
	the latest release version, which is tested and deemed working. The other 
	link 
	fetches the latest revision, which may contain bugs.
	
	\url{https://github.com/MatthijsLasure/MonteCarlo/releases/latest}
	
	\url{https://github.com/MatthijsLasure/MonteCarlo/archive/master.zip}.
	
	If git is installed on the system, following command can also be issued to 
	obtain a ready to go directory:
	
	\begin{lstlisting}[caption=Clone with git]
	git clone https://github.com/MatthijsLasure/MonteCarlo.git
	\end{lstlisting}
	
	This command will generate a MonteCarlo-directory, containing all source 
	files.
	
	\subsection{Compiling}
	Compiling is very easy.
	Inside the MonteCarlo source directory, make a directory called either 
	\verb|tmp-gnu| or \verb|tmp-intel|, and issue following command:
	
	\begin{lstlisting}[caption=Compiling the main program]
	make GNU   # To use the GFortran compiler
	make intel # To use the Intel compiler
	\end{lstlisting}
	
	The executable \verb|MonteCarloGNU| or \verb|MonteCarloOpenMP| will be 
	created.
	Both have the OpenMP standard enabled to allow for multi-threaded 
	PM6-calculations.
	
	All the utilities are compiled using the GNU compiler in similar fashion.
	Just run \verb|make| in the source directory to obtain the binaries.
	
	\section{Set-up}
	It is strongly advised to make a directory for every simulation, as to keep 
	files separated. Copy all necessary files into this directory, and make 
	sure 
	that the executable files (\verb|run.sh| and \verb|MonteCarloGNU|) have the 
	correct permissions (executable). Following files are necessary:
	
	\begin{tabular}{lll}
		File & Job & Description \\ \hline \hline
		MonteCarloGNU & General & Binary executable \\
		run.sh & General & Setup and control script \\
		config.ini & General & Configuration file \\
		box.txt & General & Input data \\
		param.txt & Lennard-Jones & Parameters for the solvent LJ calculations 
		\\
		par\_sol.txt & Lennard-Jones & Parameters for the solute LJ 
		calculations \\
		config0.ini & Minimalisation & Configuration file without solute 
		rotation \\
		BoxScale & Minimalisation & Utility for rescaling the box \\
	\end{tabular}
	
	\iffalse
	\begin{tabular}{l}
		General \\ \hline
		MonteCarloGNU \\
		run.sh \\
		box.txt \\
		config1.ini \\
	\end{tabular}
	\hspace{0.5cm}
	\begin{tabular}{l}
		Lennard-Jones \\ \hline
		param.txt \\
		par\_sol.txt \\
		\null \\
		\null \\
	\end{tabular}
	\hspace{0.5cm}
	\begin{tabular}{l}
		Minimalisation \\ \hline
		BoxScale \\
		config0.ini \\
		\null \\
		\null \\
	\end{tabular}
	\fi
	
	For a breakdown of the files and their syntax, please see chapter XX.
	
	\chapter{Getting ready: preparations}
% Getting ready: preparations
Before the actual simulation can start, some preparations have to be performed.
This consists of preparing the solute, the box and performing a minimization 
using Lennard-Jones. Finally, the box is rescaled to reflect the correct 
density.

\section{IMPORTANT NOTES}
\subsection{Floating Point Values}
If the program expects a floating point value but an integer is given (because 
the value happens to be round), it will be incorrectly read. Always make sure 
there is a point.

For example, see the solute rotation setting:\\
Wrong: \vispace\vispace 8 \vispace\vispace\vispace 1 \vispace 60 \\
Good: \vispace\vispace 8 \vispace\vispace\vispace 1 \vispace 60.0

\subsection{End Of File}
Fortran expects an empty line at the end of the file. Not complying will result 
in an incorrect read. This is very important when preparing data on a Windows 
machine. It is also advised to use the \verb|dos2unix| and \verb|checknprep| 
(included in the source code) utilities to ensure correct conventions.

\section{Box preparation}
For smaller molecules ($N < 25$), the included box can be used. It contains 30 
DMSO molecules (random placed and rotated) in a \SI{20}{\angstrom} sided box.
To generate a new box, it generally suffices to generate 6 random values per 
DMSO molecule, where the first 3 represent the position of the Centre of Mass 
(CoM), and the latter are the angles of rotation. The coordinates are relative 
to the middle of the box, so they will be in $[-r/2, +r/2]$. The angles lie in 
$[-\pi, +\pi]$. Use following syntax:

\begin{enumerate}
	\item The half length of the edge of the box, in angstrom ($r/2$);
	\item Number of DMSO molecules;
	\item The energy of the box (see note);
	\item The coordinates and angles of all molecules, one per line (X Y Z 
	$\alpha$ $\beta$ $\gamma$);
	\item A line that says \verb|SOLUTE|;
	\item The solute (see next section).
\end{enumerate}

Note: the energy of the box will be calculated during the simulation when QM 
calculations are performed (see Getting Started). When starting with 
minimisations it is sufficient to set it to $0.0$.

\section{Solute preparation}
This section can be performed with a program of choice that can handle 
molecules, such as ADF, Gaussian or Chem3D. Draw the molecule and use a quick 
minimize (if available). Afterwards, perform a good optimisation to get a good 
structure. Good results have been achieved using Gaussian and the B3LYP/6-31G* 
potential-base set combination.
When finished, obtain the cartesian coordinates from the output. Perform a PM6 
single point calculation, and obtain the energy in Hartree.

Open the box file, and look for the line that says \verb|SOLUTE|. After this 
line, write the following:

\begin{enumerate}
	\item The number of solute atoms;
	\item A description of the solute (e.g. name);
	\item All atom symbols and coordinates, one line per atom;
	\item The energy in Hartree;
	\item The number of bonds used for internal rotation;
	\item A list of bonds used for rotation (see below);
	\item An empty line.
\end{enumerate}

\subsection{Bonds for solute rotation}
This list contains bonds that can be rotated around. The syntax is the 
following:

\verb|III_III_FFFFFF|

With III being the number for the atoms of the bond. Notice that one integer 
MUST occupy 3 characters. The underscore is a character that will be ignored 
(it is recommended to use a space), 
and FFFFF is a floating point value, that is used to indicate an individual 
maximal rotation, overriding the general maximal rotation. This is optional, 
but will be written by the program in the result file.

In short, writing the following will result in unexpected results:

\verb|8 1| will read in as $(81, 0)$ instead of $(8, 1)$. Rather use 
\verb|008 001|.

\begin{lstlisting}[caption=Example of a correct box-file]
7.7838685667999998     
30
-796.46817126240921     
BOX 9
-4.93  6.07  2.42 -2.07 -1.17   1.32     
4.66 -1.67  6.16  2.76  2.64  -1.68     
...     
SOLUTE
11
Solute: (R)-2-Chloropropanoic acid
C  0.30  0.37 -0.51
H  0.50 -0.02 -1.51
Cl 1.54 -0.39  0.57
C  0.42  1.89 -0.48
H -0.31  2.33 -1.15
H  0.25  2.27  0.52
H  1.42  2.19 -0.81
C -1.08 -0.12 -0.12
O -2.09  0.12 -0.73
O -1.05 -0.85  1.01
H -1.97 -1.13  1.18
-0.16947515     
1
8   1  60.000000

\end{lstlisting}

It cannot be stressed enough that there MUST be an empty line at the end of the 
file to ensure correct reading of the data.

\section{Minimisation theory}
In the run script, there is a section for the minimisation, between the 
\verb|if false; then| en \verb|fi| statements. To run both the minimisation and 
the simulation, change the \verb|false| to \verb|true|. If you only want to run 
the minimisation, copy the first section (until the \verb|fi|) to a seperate 
run script.

The run script will perform 4 Lennard-Jones simulations with 1 million steps 
each. In between simulations the box is rescaled.

The density of the box can be calculated using formula~\ref{eqn:density}. If 
assumed that the solute is of equal volume as a DMSO molecule (which is the 
case for e.g. chloropropanoic acid), n is 31 and the density is $0,5027 
\textrm{ g cm}^{-3}$. After a rescaling the correct edge length is obtained to 
get the density of $1,09037 \textrm{ g cm}^{-3}$.\cite{Radhamma2008}
The process is done in multiple steps to allow the system to relax so no 
molecules are clipping into each other.

\begin{equation} \label{eqn:density}
\rho = \dfrac{n}{(r \si{\angstrom})^3} = \dfrac{n}{(r \si{\cdot 10^{-7} cm})^3} 
=
\dfrac{n}{(r \cdot 10^{-7})^3} \cdot \dfrac{M_{DMSO}}{N_A} \dfrac{g}{cm^3}
\end{equation}

After the LJ simulations, a long QM simulation is run to ensure the box is well 
optimized.

In the config file (\verb|config0.ini|), the parameters have been set to not 
rotate the solute (as 
this is not the actual simulation). As explained in section XX, the number of 
steps for LJ and GA are ignored as they are passed via command line arguments.

The minimazation is done in several steps in the run script. If necessary, 
change the scaling factor to obtain a bigger or smaller box.

% Getting started: the simulation

\section{Checking the output}
When the minimization is done, open the last log file and ensure that the 
energy is negative. For chloropropanoic acid, typical values are around $-800 
\textrm {kJ/mol}$.

If the energy is positive some molecules are still clipped into each other. 
This can be verified by visualisation using the ConvertDump tool and Jmol (see 
chapter 
\ref{chap:convertdump}).
In this case make another long LJ run (possibly use 5 million), again followed 
by another long GA run.

\section{Starting}
If you already did the minimisation and choose for the all-in-one package, you 
don't have to do anything.

If you are starting without minimisation, make sure it's turned off in the 
\verb|run.sh| script. Then start the script.

\subsection{Where to run?}
The program is likely to run very long. It is thereby very important to check 
how to start the program: if you run it in a SSH session, it will be terminated 
when the session closes due to network issues or a client crash.

To work around this problem, one can use the 'byobu' package. This will open a 
new shell that stays alive, even when the calling terminal is closed. 

Open a new shell using \verb|byobu bash|. To detach the shell, press CTRL-A and 
then CTRL-D. For reattaching the shell, just type \verb|byobu|.


\subsection{Start}\label{subs:start}
When you are ready to start, run following command:

\begin{lstlisting}[caption=runmin]
./run.sh &> log.txt
\end{lstlisting}

By using \verb|&> log.txt|, all messages are written to log.txt.

Congratulations! The program should be running now!
		
	\chapter{Getting into it: keeping taps on the simulation}
% Getting into it: keeping taps on the simulation

There are a few tools available to keep an eye on your simulation, something 
that is strongly recommended to do.


\section{HTOP}
Most Linux distributions come with \verb|htop| preinstalled. After startup, you 
can filter on username by pressing 'u', and apply a filter by pressing 'F4'. 

When starting a LJ simulation, 8 cores will be used by Gaussian for the 
calculation of the partial charges. Afterwards, the program will run on one 
core, which should reach 100\%. Due to the OpenMP design there will be multiple 
threads visible (as set in the run script), but only one will actually do 
something.

In the case of a QM simulation, multiple cores will be used. Currently 
(27/07/2016) 2 cores will be maxed out while the other 6 hover around 5\%. This 
is normal behaviour.

If the program is running (= is present in the list of open programs) but there 
is no CPU activity it may have locked itself. Consult the logfile and the 
troubleshooting section of this manual.

On 8-core systems, a system load of 4 to 5 is observed.

\section{check.sh}
The \verb|check.sh| script is a simple wrapper for the 'tail' command: it will 
find the latest output file and display the latest line. After sleeping 1 
second it erases the line and starts over.

Start it using the command below. Terminate using CTRL-C.
\begin{lstlisting}[caption=The check utility]
./check.sh log.txt
\end{lstlisting}

\section{MatLab}
The scripts \verb|readIO.m| and \verb|importfileFast.m| are designed to read in 
the data from the output files, and generate the appropriate graphs.

Copy the scripts to the main directory of the simulation. Open \verb|readIO.m| 
in the editor, and change the loop length to the amount of steps. Save and run 
the script.

The script will plot the accepted energy versus the step in figure 2. Following 
variables are available:

\begin{tabular}{ll}
	Variable & Description \\ \hline
	dposma & Maximal travel distance per MC step; \\
	ii & Step numbers; \\
	kans & Chance of current MC step being accepted; \\
	pSuc & Total rate of accepted steps; \\
	ratio & Rate of accepted steps since last adjust; \\
	rSo & Solvent molecule that was moved in the step; \\
	rv & Random variable for the Metropolis method; \\
	TotEng & Total energy of the current step; \\
	TotEng\_old & Total energy of the last accepted step. \\
\end{tabular}

\section{ConvertDump}
The convertdump utility can convert the box- and dump-files into XYZ-files that 
can be read using Jmol.

To convert a box to a XYZ file without displaying hydrogen atoms (to save space 
and time), use the command below.

\begin{lstlisting}[caption=Convert a box to a XYZ file]
$ ./convertDump -m 1 -b box.5.out -o box.5.xyz
+------------------+
| Converter for MC |
+------------------+
Using internal coordinates for DMSO.
\end{lstlisting}

For the full documentation, please see chapter \ref{chap:convertdump}.

\section{Analysis of the conformers}
Using the scripts \verb|post.sh| and \verb|cenergy.m| you can get an overview 
of the different conformers. For this, the utility \verb|calcDihedral| and the 
\verb|post.sh| script must be present in the main directory. Edit the 
post-script and insert the atoms for the dihedral angle. Run it, with as 
argument the logfile. This will result in a few files: acc.txt, energy.txt and 
hoeken.txt. Put these in the same directory as the cenergy-script, and run it 
in matlab.

NOTE: currently, only single bond rotations are supported.
	
\chapter{Getting out of the woods: processing the results}
% Getting out of the woods: processing the results

The simulation in itself will just create some configurations. Now we need to 
select some configurations, and prepare them for the VCD calculations.

To facilitate the process, scripts are provided to take on most of the work. 
The \verb|sim2ext.sh| script will extract (some) accepted configurations from 
the 
pool. After converting these to XYZ using the ConvertDump utility, the 
\verb|xyz2freq.sh| script turns them into Gaussian input files for 
optimalization and frequency calculations.
When done, the spectra are extracted with Amphidromus and plotted with MatLab.

\section{Extracting configurations}

Run the \verb|sim2ext.sh| script in the simulation directory with following 
arguments: the logfile, the destination directory (optional) and the interval 
(optional). It will take every nth accepted configuration (with n the interval)
and place it in the destination directory.

\begin{lstlisting}[caption=The sim2ext script]
$ ./sim2ext.sh totlog.txt test/
Files read: 5436
Box accept: 3514 (64.6400 %)
Box copy:   140
\end{lstlisting}

\section{Configurations to XYZ}\label{sec:conf2xyz}

Place the ConvertDump utility in the sample directory, and run it on every 
configuration.

\begin{lstlisting}[caption=Using the ConvertDump utility to make XYZ's]
for file in *; do ./ConvertDump -m 3 -b $file -o ${file%.out}.xyz; done
\end{lstlisting}

Please take note that the mode (the -m flag) is 3, rather than 1. This mode 
will sort the solvent molecules from closest to the solute to furthest. 
The output file will have the same name but with the XYZ extension.

\section{Optimalization and frequency calculations}
Put the \verb|xyz2freq.sh| script in the same directory as the XYZ files. Then 
issue the following command, replacing XX with the number of solute atoms.

\begin{lstlisting}[caption=XYZ to COM]
for file in *; do ./xyz2opt.sh $file XX; done
\end{lstlisting}

This will generate the input COM files for Gaussian. It will use 20 processing 
cores with 12 GB of RAM. The calculations are on the B3LYP/6-31G(d) level. 
The optimalization is limited to 50 cycles, as it sometimes does not converge.
All solvent molecules are frozen.

The frequency calculation happens in deuterated solvent.

All calculations are performed at 293.0 K.

\null

If you want to change these settings, you can edit the script. The CPU and RAM 
settings may have a large performance impact if overestimated.

Below is the standard header.
\begin{verbatim}
%mem=12GB
%nproc=20
%chk=hey.chk
#P b3lyp/6-31g(d) opt(maxcycles=50) freq=(vcd,readisotopes)

opt + freq

0 1
Solute
Solvent

293. 1. 1.
Isotopes

\end{verbatim}

Please check the files before submission to ensure that the syntax is correct. 

If the optimalization hits the cycle limit the calculation will terminate.
You can either restart the optimalization (with the last results) or skip and 
go straight to the frequency calculations.

\section{Spectrum}
The Gaussian log files have to be processed into readable spectra, and then 
added together into one final spectrum. This can be done using Amphidromus (not 
included). 
Execute the program inside the directory with the log files. The program will 
report any errors that may occur. If imaginary frequencies are present the 
configuration is ignored. This can be overridden with the Amphidromus 
configuration file.

Using \verb|doImport.m| script with the \verb|importfileAmphi.m| helpscript the 
spectra will be added together into the final spectrum and displayed on figure 
4.
