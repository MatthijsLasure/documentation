% Getting out of the woods: processing the results

The simulation in itself will just create some configurations. Now we need to 
select some configurations, and prepare them for the VCD calculations.

To facilitate the process, scripts are provided to take on most of the work. 
The \verb|sim2ext.sh| script will extract (some) accepted configurations from 
the 
pool. After converting these to XYZ using the ConvertDump utility, the 
\verb|xyz2freq.sh| script turns them into Gaussian input files for 
optimalization and frequency calculations.
When done, the spectra are extracted with Amphidromus and plotted with MatLab.

\section{Extracting configurations}

Run the \verb|sim2ext.sh| script in the simulation directory with following 
arguments: the logfile, the destination directory (optional) and the interval 
(optional). It will take every nth accepted configuration (with n the interval)
and place it in the destination directory.

\begin{lstlisting}[caption=The sim2ext script]
$ ./sim2ext.sh totlog.txt test/
Files read: 5436
Box accept: 3514 (64.6400 %)
Box copy:   140
\end{lstlisting}

\section{Configurations to XYZ}

Place the ConvertDump utility in the sample directory, and run it on every 
configuration.

\begin{lstlisting}[caption=Using the ConvertDump utility to make XYZ's]
for file in *; do ./ConvertDump -m 3 -b $file -o ${file%.out}.xyz; done
\end{lstlisting}

Please take note that the mode (the -m flag) is 3, rather than 1. This mode 
will sort the solvent molecules from closest to the solute to furthest. 
The output file will have the same name but with the XYZ extension.

\section{Optimalization and frequency calculations}
Put the \verb|xyz2freq.sh| script in the same directory as the XYZ files. Then 
issue the following command, replacing XX with the number of solute atoms.

\begin{lstlisting}[caption=XYZ to COM]
for file in *; do ./xyz2opt.sh $file XX; done
\end{lstlisting}

This will generate the input COM files for Gaussian. It will use 20 processing 
cores with 12 GB of RAM. The calculations are on the B3LYP/6-31G(d) level. 
The optimalization is limited to 50 cycles, as it sometimes does not converge.
All solvent molecules are frozen.

The frequency calculation happens in deuterated solvent.

All calculations are performed at 293.0 K.

\null

If you want to change these settings, you can edit the script. The CPU and RAM 
settings may have a large performance impact if overestimated.

Below is the standard header.
\begin{verbatim}
%mem=12GB
%nproc=20
%chk=hey.chk
#P b3lyp/6-31g(d) opt(maxcycles=50) freq=(vcd,readisotopes)

opt + freq

0 1
Solute
Solvent

293. 1. 1.
Isotopes

\end{verbatim}

Please check the files before submission to ensure that the syntax is correct. 

If the optimalization hits the cycle limit the calculation will terminate.
You can either restart the optimalization (with the last results) or skip and 
go straight to the frequency calculations.

\section{Spectrum}
The Gaussian log files have to be processed into readable spectra, and then 
added together into one final spectrum. This can be done using Amphidromus (not 
included). 
Execute the program inside the directory with the log files. The program will 
report any errors that may occur. If imaginary frequencies are present the 
configuration is ignored. This can be overridden with the Amphidromus 
configuration file.

Using \verb|doImport.m| script with the \verb|importfileAmphi.m| helpscript the 
spectra will be added together into the final spectrum and displayed on figure 
4.