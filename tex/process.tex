% Getting out of the woods: processing the results

The simulation in itself will just create some configurations. Now we need to 
select some configurations, and prepare them for the VCD calculations.

To facilitate the process, scripts are provided to take on most of the work. 
The \verb|sim2ext.sh| script will extract (some) accepted configurations from 
the 
pool. After converting these to XYZ using the ConvertDump utility, the 
\verb|XYZ2opt.sh| script turns them into Gaussian input files for 
optimalization. When done, the \verb|opt2freq.sh| script makes the final input 
files for the frequency calculations.

\section{Extracting configurations}

Run the \verb|sim2ext.sh| script in the simulation directory with following 
arguments: the logfile, the destination directory (optional) and the interval 
(optional). It will take every nth accepted configuration (with n the interval)
and place it in the destination directory.

\begin{lstlisting}[caption=The sim2ext script]
$ ./sim2ext.sh totlog.txt test/
Files read: 5436
Box accept: 3514 (64.6400 %)
Box copy:   140
\end{lstlisting}

\section{Configurations to XYZ}

Place the ConvertDump utility in the sample directory, and run it on every 
configuration.

\begin{lstlisting}[caption=Using the ConvertDump utility to make XYZ's]
for file in *; do ./ConvertDump -m 3 -b $file -o ${file%.out}.xyz
\end{lstlisting}

Please take note that the mode (the -m flag) is 3, rather than 1. This mode 
will sort the solvent molecules from closest to the solute to furthest.