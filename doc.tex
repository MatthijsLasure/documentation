\documentclass[a4paper,fleqn]{report}

\usepackage{thesisStyle}

\begin{document}
	
	\chapter{User's manual}

	
	This document will explain the steps needed to obtain a successfull VCD-spectrum using the DMSO Monte-Carlo simulation.
	
	Following steps are neccesary:
	\begin{enumerate}
		\setlength{\itemsep}{-1pt}
		\setlength{\parsep}{-1pt}
		\item Installation of neccesary programs and configuration
		\item Preparation of the simulation
		\item Minimization and density control
		\item Simulation
		\item Checks of the results
		\item Processing of the results for QM calculations
		\item Spectra calculations
		\item Final spectrum
	\end{enumerate}
	
	\section{Environment preparation}
	In order to run, both Gaussian 09 and MOPAC are necessary. Gaussian is used for the initial charge calculations, and both versions A.02 and D.01 have been tested.
	MOPAC2016 is used for the PM6 energy calculations. Version 16.175L is known to work.
	
	For the installation of Gaussian, please refer to the manual. As long as the path is added to the \verb|PATH|-variable it should work.
	
	
	
\end{document}